\section{Facial Detection}
\label{sec:human}

%----------------------------------------------------------------------
%           FACE DETECTION
%----------------------------------------------------------------------
%\subsection{Face Detection}

Haar Cascade Classifiers is the method chosen to detect human faces. 

\subsection{Indoors Facial Detection: Environment}

The indoor evaluation environment was a 12x20 ft room. The room was lit with two overhead, bare T5 fluorescent bulbs in parallel, outputting approximately 3500K neutral white. The primary evaluation distance was 4 ft, however, 8 ft and 10 ft were also evaluated for scale invariance.

\begin{figure}[H]
    \centering
    \makebox[\textwidth][c]{\includegraphics[width=0.75\textwidth]{pics/eval_in_dist.jpg}}
    \caption{Indoor evaluation environment. The prototype is on the right edge of the TV stand. The white plastic bag is approximately 4 ft from it, and Nelson's position is approximately 8 ft away.}
    \label{fig:dist}
\end{figure}


\subsection{Indoors Facial Detection: Head Wear}

\begin{figure}[H]
    \centering
    \begin{subfigure}[b]{0.31\textwidth}
        \includegraphics[width=\textwidth]{pics/face1.jpg}
        \caption{Sunglasses.}
        \label{fig:face1}
    \end{subfigure}
    ~ %add desired spacing between images, e. g. ~, \quad, \qquad, \hfill etc. 
      %(or a blank line to force the subfigure onto a new line)
    \begin{subfigure}[b]{0.31\textwidth}
        \includegraphics[width=\textwidth]{pics/face2.jpg}
        \caption{Hoodie and sunglasses.}
        \label{fig:face2}
    \end{subfigure}
    \begin{subfigure}[b]{0.31\textwidth}
        \includegraphics[width=\textwidth]{pics/face3.jpg}
        \caption{Baseball cap.}
        \label{fig:face3}
    \end{subfigure}
    \begin{subfigure}[b]{0.31\textwidth}
        \includegraphics[width=\textwidth]{pics/face4.jpg}
        \caption{Paratrooper beret.}
        \label{fig:face4}
    \end{subfigure}
    ~ %add desired spacing between images, e. g. ~, \quad, \qquad, \hfill etc. 
      %(or a blank line to force the subfigure onto a new line)
    \begin{subfigure}[b]{0.31\textwidth}
        \includegraphics[width=\textwidth]{pics/face5.jpg}
        \caption{Advanced Combat Helmet.}
        \label{fig:face5}
    \end{subfigure}
    \begin{subfigure}[b]{0.31\textwidth}
        \includegraphics[width=\textwidth]{pics/face6.jpg}
        \caption{Shemagh.}
        \label{fig:face6}
    \end{subfigure}
    \caption{Indoor facial detection evaluated through various head wear at approximately 3-4 ft distance.}\label{fig:face_indoors}
\end{figure}

\subsection{Indoors Facial Detection: Distance}

Figure \ref{fig:invariant} demonstrates that the system is able to detect human faces at 3 ft., 8 ft., and 10 ft. The dog in the frame was disregarded as this specific evaluation only searched for human faces.

\begin{figure}[H]
    \centering
    \makebox[\textwidth][c]{\includegraphics[width=0.70\textwidth]{pics/eval_in_invariant.jpg}}
    \caption{Positive detection for three humans at different distances.}
    \label{fig:invariant}
\end{figure}

%\subsubsection{Daytime Outdoors Facial Detection}

%\subsubsection{Nighttime Outdoors Facial Detection}
\subsection{Nighttime Facial Detection}

Low-ambient lighting conditions proved to be a formidable and admittedly underestimated challenge. The success rate for human facial detection remains low, hovering around 20\% with the 3-IR LED Night Vision board activated. The chance for a positive detection increases when the distance from the Night Vision board decreases. At approximately 3 ft., the success rate increases to approximately 40\%.

Plans to increase success rate includes:
\begin{singlespace}
\begin{enumerate}
    \item Modifying the intensity of the IR LEDs.
    \item Increasing the numbers of IR LEDs to the Night Vision board.
    \item Increasing the numbers of Night Vision boards.
\end{enumerate}
\end{singlespace}

\begin{figure}[H]
    \centering
    \makebox[\textwidth][c]{\includegraphics[width=1\textwidth]{pics/eval_out_nightIR_face2.jpg}}
    \caption{Positive facial detection at night, IR LEDs active.}
    \label{fig:nightRon1}
\end{figure}

%----------------------------------------------------------------------
%           Motion DETECTION
%----------------------------------------------------------------------
%\subsection{Motion Detection}


%----------------------------------------------------------------------
%           SILHOUETTE: HOG
%----------------------------------------------------------------------
%\subsection{Human Silhouette Detection}

%\subsubsection{Preliminary Indoors Silhouette Detection}

%\subsubsection{Daytime Outdoors Silhouette Detection}

%\subsubsection{Nighttime Outdoors Silhouette Detection}