\section*{Calculations}
\label{sec:calc}
% CHANGE GB TO 1073741824 BYTES!!!

\subsection{Storage}

Storage requirements for 24 hours of 720p video data (.mp4):
\begin{equation*}
\begin{split}
Continual & = \left(\frac{24 \ hr}{1 \ day}\right)\left(\frac{60 \ min}{1 \ hr}\right)\left(\frac{60 \ sec}{ 1 \ min}\right)\left(\frac{17 \ mbit}{1 \ sec}\right)\left(\frac{1 \ byte}{8 \ bit}\right)\left(\frac{1 \ GB}{1737441824 \ byte}\right) \\
& = 170.99 \ \frac{GB}{day}
\end{split}
\end{equation*}

By limiting the active recording to moments where objects are detected, this storage requirement can be significantly lowered. Taking a theoretical family of 4 and a single dog, we can calculate for a minimum duration that the RPi2 is expected to record in 24 hours: 

\begin{table}[H]
\centering
\caption{Known \glspl{sighting} during an average 24 hour day for a front door camera.}
\label{floor_table}
\begin{tabular}{@{}lcccc@{}}
\toprule
Type     & Qty & Duration (min) & Sighting/24 hr & Total  \\ \midrule
Person   & 4   & 1              & 2               & 8      \\
Dog      & 1   & 1              & 2               & 2      \\
Delivery & 2   & 1              & 1               & 2      \\ \midrule
         &     &                &                 & 12 min
\end{tabular}
\end{table}

With a minimum recording duration of 12 minutes based on the expected sightings (detected objects), the floor data requirements per 24 hours becomes:

\begin{equation*}
\begin{split}
Floor \ & = \left(\frac{12 \ min}{1 \ day}\right)\left(\frac{60 \ sec}{ 1 \ min}\right)\left(\frac{17 \ Mbit}{1 \ sec}\right)\left(\frac{1 \ byte}{8 \ bit}\right)\left(\frac{1 \ GB}{1737441824 \ byte}\right) \\
& = 0.000014254 \ \frac{GB}{day}\Rightarrow 0.014254 \ \frac{MB}{day} \Rightarrow 14.254 \ \frac{KB}{day}
\end{split}
\end{equation*}

\subsection{Analog Light Sensor}

\begin{equation*}
\begin{split}
    \log\left(I_{O}\right) &= mx + b \\
    I_{O} &= m\log\left(x\right) + b \\
    I_{O} &= 10\log\left(E_{v}\right)\left[l_{x}\right]
\end{split}
\end{equation*}

The sensor has an output current range of $0\mu A$--$50\mu A$ between $1$ lux--$100000$ lux. The breakout board converts this current into a voltage using a $68K\Omega$ resistor, giving a voltage output of $0V$--$3.4V$. Given the use-case of the light sensor, the $3.4V$ can be approximated to $3.3V$, which is a suitable voltage reference for the RPi2. With the output current range approximated to a voltage range of $0V$--$3.3V$ between $1$ lux--$100000$ lux,

\begin{equation*}
\begin{split}
    log_{i} &= \log\left(l_{x \ MAX} - l_{x \ MIN}\right) \\
    &= \log\left(100000\right) = 5 \\
\end{split}
\end{equation*}

To convert the analog value to a digital value, an 8-bit ADC is used:
\begin{equation*}
    ADC = \frac{log_{i}}{1024 \ steps} = 0.00488
\end{equation*}

Finally, the formula to obtain the final lux output, $l_{xo}$, on the RPi2:

\begin{equation*}
%\begin{split}
    l_{xo} = 10^{raw_{i} \times ADC} = 10^{raw_{i} \times 0.00488}
%\end{split}
\end{equation*}

where $raw_{i}$ is the raw value received by the ADC connected to the RPI2's GPIO.

%\newpage

%\begin{multicols}{2}
%\begin{onehalfspacing}
%\printglossary
%\end{onehalfspacing}
%\end{multicols}