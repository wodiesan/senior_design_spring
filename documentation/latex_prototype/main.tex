\documentclass[12pt]{article}
%\usepackage{fontspec}
%\usepackage[utf8]{inputenc}

\usepackage[margin=1in]{geometry}

\usepackage{graphicx}
\usepackage{amsmath}
\usepackage{float}
\usepackage{xfrac}
%\usepackage{subfig}
\usepackage{subcaption}
\captionsetup{belowskip=12pt,aboveskip=4pt}
\usepackage{booktabs}
\usepackage{enumitem}
\usepackage{csquotes}

\usepackage{amsmath}
\usepackage{mathptmx}
\usepackage{helvet}
\renewcommand{\familydefault}{\sfdefault}
\usepackage{varwidth}
\usepackage{verbatim}

\newenvironment{centerverbatim}{%
  \par
  \centering
  \varwidth{\linewidth}%
  \verbatim
}{%
  \endverbatim
  \endvarwidth
  \par
}

\usepackage{gensymb}
\usepackage{setspace}
\usepackage{booktabs}
\usepackage{hyperref}
\usepackage{tikz}
\usetikzlibrary{trees}

\usepackage{listings}
\usepackage{color}
\definecolor{rp}{rgb}{0.47, 0.32, 0.66}
\definecolor{rb}{rgb}{0.25, 0.41, 0.88}
\definecolor{tp}{rgb}{0.4, 0.01, 0.24}
\definecolor{lightgrey}{rgb}{0.9,0.9,0.9}
\definecolor{darkgreen}{rgb}{0,0.6,0}

%---------------------------------------------------------------------
%           FOOTER
%---------------------------------------------------------------------
\usepackage{fancyhdr}
\pagestyle{fancy}
\cfoot{Confidential $\mid$ ELEC4500-2016-PROTO Rev. 1}
\rfoot{\thepage}

\usepackage{multicol} % Required for splitting text into multiple columns
\setlength{\columnsep}{1cm}
\usepackage[toc]{glossaries}
\usepackage[protrusion=true,expansion=true]{microtype} % Improves spacing
%\usepackage[bf,sf,center]{titlesec} % Required for modifying section titles - bold, sans-serif, centered
%\usepackage{fancyhdr} % Required for modifying headers and footers
%\usepackage{subfiles}

\usepackage[toc,page]{appendix}

%\makeglossaries
%%\begin{multicols}{2}

\newglossaryentry{API}
{
    name=API,
    description={Application Programming Interface is set of tools--usually in the form of a library--to build software-based components and to integrate various technologies.}
}

\newglossaryentry{Blob}
{
    name=blob,
    description={Computer vision method that detects consistent regions within other regions.}
}

\newglossaryentry{Dropbox}
{
    name=Dropbox,
    description={Cloud-based file hosting service that syncs similar-named directories across systems with the configured app.}
}

\newglossaryentry{Google Drive}
{
    name=Google Drive,
    description={Cloud-based file storage and sync service from Google used in this project to draft documents.}
}

\newglossaryentry{GitHub}
{
    name=GitHub,
    description={Online hosting service for git-based source code management and revision control.}
}

\newglossaryentry{Haar}
{
    name=Haar,
    description={Object detection method by P. Viola and M. Jones, \emph{Rapid Object Detection using a Boosted Cascade of Simple Features}.}
}

\newglossaryentry{LAMP}
{
    name=LAMP,
    description={Standard web service stack consisting of open source technologies such as Linux, Apache, MySQL, and Python.}
}

\newglossaryentry{livestream}
{
    name=livestream,
    description={Real-time Internet transmission of an event.}
}

\newglossaryentry{MySQL}
{
    name=MySQL,
    description={My Structured Query Language is a relational database management system released on a GPL v.2 license.}
}

\newglossaryentry{detection}
{
    name=detection,
    description={Techniques to locate objects in still images or video sequences.}
}

\newglossaryentry{recognition}
{
    name=recognition,
    description={Techniques to identify or verify objects in still images or video sequences through a stored database of known faces.}
}

\newglossaryentry{tracking}
{
    name=tracking,
    description={Techniques to locate moving objects in video sequences.}
}

\newglossaryentry{OpenCV}
{
    name=OpenCV,
    description={Open Source Computer Vision is a computer vision library with an open-source BSD license developed at Intel.}
}

\newglossaryentry{open source}
{
    name=open source,
    description={Software and hardware that is released to the public with a free license, allowing for universal access, distribution, and collaborative improvements.}
}

\newglossaryentry{Pi}
{
    name=Raspberry Pi,
    description={The Raspberry Pi is a series of sub-\$50 single-board computers designed for educational usage. For the purposes of this document, all references--unless specified otherwise--refers to the Raspberry Pi 2 Model B.}
}

\newglossaryentry{Plotly}
{
    name=Plotly,
    description={Online data visualization tool with scientific graphing libraries for Python.}
}

\newglossaryentry{REST}
{
    name=REST,
    description={Representational State Transfer is an World Wide Web architectural style for client--server communications.}
}

\newglossaryentry{sighting}
{
    name=sighting,
    description={In this case, each sighting is an interval of time where the system has detected an object.}
}
 
\newglossaryentry{ShareLaTeX}
{
    name=ShareLaTeX,
    description={Online \LaTeX \ editor with revision control and real-time collaboration.}
}
 
\newglossaryentry{SimpleCV}
{
    name=SimpleCV,
    description={Simple Computer Vision is a simplified version of OpenCV. Like OpenCV, SimpleCV is open source.}
}

\newglossaryentry{sprint}
{
    name=sprint,
    description={A basic unit of development in Scrum methodology. It is a timeboxed iteration that contains then existing backlog, current scope of work, and estimated commitment from each resource.}
}

\newglossaryentry{stand-up}
{
    name=stand-up,
    description={A timeboxed meeting at the beginning of the team's workday where each resource gives an update on the events since the last stand-up, the plans for the day, and announces any roadblocks.}
}

\newglossaryentry{VNC}
{
    name=VNC,
    description={Virtual Computing Network is a system used to remotely control another computer.}
}

\newglossaryentry{WebIDE}
{
    name=WebIDE,
    description={Adafruit's web-based Integrated Development Environment designed for the Raspberry Pi.}
}

%\end{multicols}

\begin{document}\thispagestyle{empty}
\pagenumbering{gobble}
\newcommand*{\titleTMB}{\begingroup% Three Men in a Boat
\newlength{\drop}
\drop=0.1\textheight
\centering

\begin{flushright}
{\small\scshape Nelson Bamford\\Sze ``Ron'' Chau\\Ariel Martinez}\\[\baselineskip]
\end{flushright}

\settowidth{\unitlength}{\LARGE THE BOOK OF CONUNDRUMS}
%\vspace*{\baselineskip}

\vspace{1in}

\rule{\unitlength}{1.6pt}\vspace*{-\baselineskip}\vspace*{2pt}
\rule{\unitlength}{0.4pt}\\[\baselineskip]
%{\Large Attuning or Overcoming Inherent Nature}\\[\baselineskip]
{\large ENGINEERING PROTOTYPE}\\[\baselineskip]
{\itshape --$for$--}\\[0.3\baselineskip]
{\itshape a Home Surveillance System}\\[0.2\baselineskip]
\rule{\unitlength}{0.4pt}\vspace*{-\baselineskip}\vspace{3.2pt}
\rule{\unitlength}{1.6pt}\\[\baselineskip]
{\LARGE\scshape wentworth institute of technology}\par
{\large\scshape ELEC 4500: SENIOR ELECTRONIC DESIGN I}\par
%{\large\scshape the publisher}\\[\baselineskip]
\vfill
{\small\scshape Instructor: Professor Grenquist\\Submission: Rev. 1, 16MAR2016}\par
\vspace*{\drop}
%\thispagestyle{empty}
\endgroup}
\titleTMB
\newpage
\doublespacing


%----------------------------------------------------------------------
% Per S. Grenquist 2013 Publication RE: Senior Design Prototype Phase
%----------------------------------------------------------------------
%       Prototype Manufacturing Phase 
% After having all of their engineering calculations examined and
% certified, the students then start to build their prototype and
% design their production methods. 

%       Prototype 
% The culmination of the prototype Manufacturing phase is the 
% presentation of a working prototype. 

%       Production Design 
% The students also must provide a detailed description of the
% production process both for the prototype, as well as for larger
% production runs.

%      *Project Development*
% The Project Development section of each of the four phases of the
% BDS is used to assess how well the student group has progressed
% through the development of the project. The question asked at each
% of the developmental phases is whether the student design group is
% ahead of schedule and below budget. They're great for that section
% of the phase is dependent on how they answer that question. It is
% common to all of the four phases included in the PBDS.*

%----------------------------------------------------------------------
% WORD BANK
%----------------------------------------------------------------------
% eludicate

%----------------------------------------------------------------------
% Abstract and Table of Contents
%----------------------------------------------------------------------
\begin{abstract}\thispagestyle{empty}

A technical proof of concept for a lightweight, customizable, and scalable home surveillance system is presented in this documentation. This phase primarily of evaluating several object detection methods--in day and low ambient lighting conditions--for humans and vehicles. 
%Technical specifications, i.e. calculations, breadboard diagrams, schematics, PCB layout design, and program flowcharts for a Raspberry Pi 2-based home surveillance system is outlined in this document. This system is designed to allow a homeowner to record digital video--through low-ambient lighting environments--when an adjustable motion threshold is surpassed. The system will include human and vehicle detection features. Recorded videos will be cataloged by the object type that activated the camera, i.e. human, vehicle, and others. This proposed functionality will allow a homeowner to quickly access their database of video data through categorical ``bookmarks'' based on metadata such as date, time, and type of object. The result is a lightweight, customizable, and scalable home surveillance system that remains inexpensive for storage. 

\end{abstract}

\newpage

\tableofcontents
\listoffigures
\listoftables

\newpage

%---------------------------------------------------------------------
% Intro
%---------------------------------------------------------------------
\pagenumbering{arabic}

\addcontentsline{toc}{section}{\nameref{sec:intro}}
\section{Introduction}
\label{sec:intro}

The Home Surveillance System is divided into two subsystems: a Raspberry Pi 2 with a camera module that performs computer vision and a night vision circuit comprised of an ambient light sensor and an array of IR LEDs. These two subsystems operate separately, allowing the user to be selective in their home monitoring configuration. %While custom enclosures are being developed that will allow both subsystems to couple

The Home Surveillance System--at its current iteration--is capable of detecting human through their frontal and side facial features. Motion detection runs continuously and acts as the idling state. Once motion is detected in the frame, the system isolates the sections. Finally, it seeks out human faces in the bounded through Haar-like features--a process pioneered by Viola-Jones \cite{ViolaJones}.

\begin{figure}[H]
    \centering
    \makebox[\textwidth][c]{\includegraphics[width=1\textwidth]{pics/eval_in_remote.jpg}}
    \caption{Remote access to the livestream through VNC.}
    \label{fig:remote}
\end{figure}

The user is able to view a livestream of the camera through an encrypted VNC connection, allowing for secure access within or without the home network. Figure \ref{fig:remote}, shows a screen shot of the proof of concept's livestream. This was taken from A. Martinez's laptop, which was connected to the RPi2 from an outside network.



\newpage
\addcontentsline{toc}{section}{\nameref{sec:creds}}
\section{Prototype Specifications and Credentials}
\label{sec:creds}

\begin{table}[H]
\centering
\caption{Software utilized within prototype.}
\label{req}
\begin{tabular}{@{}lllll@{}}
\toprule
Technology & Version   & Type                              & License                     \\ \midrule
Raspbian   & Jessie    & Raspberry Pi Linux OS Distro      & GPL v2                      \\
Python     & 2.7       & General Purpose Language          & Python Software Foundation  \\
OpenCV     & 3.0.0     & Computer Vision Library           & BSD                         \\
TightVNC   & 1.3.9-6.5 & Virtual Computing Network         & GPL v2                      \\
Weaved     & 1.3-02    & Network Managing Platform         & GPL v2                      \\\bottomrule
\end{tabular}
\end{table}

\begin{table}[H]
\centering
\caption{Credentials to the prototype system.}
\label{credentials}
\begin{tabular}{@{}ll@{}}
\toprule
Name        & Type                \\ \midrule
scrying     & RPi2 Hostname       \\
constantine & RPi2 Username       \\
cv          & Virtual Environment \\\bottomrule
\end{tabular}
\end{table}

\newpage
\addcontentsline{toc}{section}{\nameref{sec:pcb}}
\section{Night Vision}
\label{sec:pcb}

\begin{figure}[H]
    \centering
    \makebox[\textwidth][c]{\includegraphics[width=1\textwidth]{pics/pcbs.jpg}}
    \caption{Night Vision subsystem printed circuit boards.}
    \label{fig:pcbs}
\end{figure}

Figure \ref{fig:pcbs} shows the two custom PCBs for the IR LED arrays. These boards are completely separate from the main Computer Vision subsystem. This allows the user to position the Night Vision boards at locations of interest. For example, the user could position the Computer Vision subsystem above their front door and position one or more Night Vision subsystems near their vehicle or walkway. The flexibility of the Night Vision subsystem allows for greater customization to the end user, who can decide to either on-board, off-board, or not utilize the boards based on their needs.  

\newpage
\addcontentsline{toc}{section}{\nameref{sec:human}}
\section{Facial Detection}
\label{sec:human}

%----------------------------------------------------------------------
%           FACE DETECTION
%----------------------------------------------------------------------
%\subsection{Face Detection}

Haar Cascade Classifiers is the method chosen to detect human faces. 

\subsection{Indoors Facial Detection: Environment}

The indoor evaluation environment was a 12x20 ft room. The room was lit with two overhead, bare T5 fluorescent bulbs in parallel, outputting approximately 3500K neutral white. The primary evaluation distance was 4 ft, however, 8 ft and 10 ft were also evaluated for scale invariance.

\begin{figure}[H]
    \centering
    \makebox[\textwidth][c]{\includegraphics[width=0.75\textwidth]{pics/eval_in_dist.jpg}}
    \caption{Indoor evaluation environment. The prototype is on the right edge of the TV stand. The white plastic bag is approximately 4 ft from it, and Nelson's position is approximately 8 ft away.}
    \label{fig:dist}
\end{figure}


\subsection{Indoors Facial Detection: Head Wear}

\begin{figure}[H]
    \centering
    \begin{subfigure}[b]{0.31\textwidth}
        \includegraphics[width=\textwidth]{pics/face1.jpg}
        \caption{Sunglasses.}
        \label{fig:face1}
    \end{subfigure}
    ~ %add desired spacing between images, e. g. ~, \quad, \qquad, \hfill etc. 
      %(or a blank line to force the subfigure onto a new line)
    \begin{subfigure}[b]{0.31\textwidth}
        \includegraphics[width=\textwidth]{pics/face2.jpg}
        \caption{Hoodie and sunglasses.}
        \label{fig:face2}
    \end{subfigure}
    \begin{subfigure}[b]{0.31\textwidth}
        \includegraphics[width=\textwidth]{pics/face3.jpg}
        \caption{Baseball cap.}
        \label{fig:face3}
    \end{subfigure}
    \begin{subfigure}[b]{0.31\textwidth}
        \includegraphics[width=\textwidth]{pics/face4.jpg}
        \caption{Paratrooper beret.}
        \label{fig:face4}
    \end{subfigure}
    ~ %add desired spacing between images, e. g. ~, \quad, \qquad, \hfill etc. 
      %(or a blank line to force the subfigure onto a new line)
    \begin{subfigure}[b]{0.31\textwidth}
        \includegraphics[width=\textwidth]{pics/face5.jpg}
        \caption{Advanced Combat Helmet.}
        \label{fig:face5}
    \end{subfigure}
    \begin{subfigure}[b]{0.31\textwidth}
        \includegraphics[width=\textwidth]{pics/face6.jpg}
        \caption{Shemagh.}
        \label{fig:face6}
    \end{subfigure}
    \caption{Indoor facial detection evaluated through various head wear at approximately 3-4 ft distance.}\label{fig:face_indoors}
\end{figure}

\subsection{Indoors Facial Detection: Distance}

Figure \ref{fig:invariant} demonstrates that the system is able to detect human faces at 3 ft., 8 ft., and 10 ft. The dog in the frame was disregarded as this specific evaluation only searched for human faces.

\begin{figure}[H]
    \centering
    \makebox[\textwidth][c]{\includegraphics[width=0.70\textwidth]{pics/eval_in_invariant.jpg}}
    \caption{Positive detection for three humans at different distances.}
    \label{fig:invariant}
\end{figure}

%\subsubsection{Daytime Outdoors Facial Detection}

%\subsubsection{Nighttime Outdoors Facial Detection}
\subsection{Nighttime Facial Detection}

Low-ambient lighting conditions proved to be a formidable and admittedly underestimated challenge. The success rate for human facial detection remains low, hovering around 20\% with the 3-IR LED Night Vision board activated. The chance for a positive detection increases when the distance from the Night Vision board decreases. At approximately 3 ft., the success rate increases to approximately 40\%.

Plans to increase success rate includes:
\begin{singlespace}
\begin{enumerate}
    \item Modifying the intensity of the IR LEDs.
    \item Increasing the numbers of IR LEDs to the Night Vision board.
    \item Increasing the numbers of Night Vision boards.
\end{enumerate}
\end{singlespace}

\begin{figure}[H]
    \centering
    \makebox[\textwidth][c]{\includegraphics[width=1\textwidth]{pics/eval_out_nightIR_face2.jpg}}
    \caption{Positive facial detection at night, IR LEDs active.}
    \label{fig:nightRon1}
\end{figure}

%----------------------------------------------------------------------
%           Motion DETECTION
%----------------------------------------------------------------------
%\subsection{Motion Detection}


%----------------------------------------------------------------------
%           SILHOUETTE: HOG
%----------------------------------------------------------------------
%\subsection{Human Silhouette Detection}

%\subsubsection{Preliminary Indoors Silhouette Detection}

%\subsubsection{Daytime Outdoors Silhouette Detection}

%\subsubsection{Nighttime Outdoors Silhouette Detection}

\newpage
\addcontentsline{toc}{section}{\nameref{sec:enclosure}}
\section{Enclosures}
\label{sec:enclosure}

During the Proposal Phase the design included a MobileApp Systems Proto Armour Enclosure for the RPi2. During cost analysis, it was quickly concluded that--while beneficial at the proof of concept stage--this component would not meet the requirements for a production-level product. The choice of the Proto Armour was based on the enclosure's robust aircraft-grade aluminum, support for the camera module, and numerous mounting holes at each side of the case. The Proto Armour, however, wasn't designed to be weatherproof. In addition, the cost per enclosure at \$45 per enclosure represents close to 30\% of the direct materials cost per unit. This high cost disqualified the Proto Armour Enclosure for purposes beyond proof of concept designs.

\begin{figure}[H]
    \centering
    \begin{subfigure}[b]{0.475\textwidth}
        \includegraphics[width=\textwidth]{pics/Case_Proto1.jpg}
        \caption{External view.}
        \label{fig:ProtoCase1}
    \end{subfigure}
    ~ %add desired spacing between images, e. g. ~, \quad, \qquad, \hfill etc. 
      %(or a blank line to force the subfigure onto a new line)
    \begin{subfigure}[b]{0.475\textwidth}
        \includegraphics[width=\textwidth]{pics/Case_Proto2.jpg}
        \caption{Internal view.}
        \label{fig:ProtoCase2}
    \end{subfigure}
    \caption{Proto Armour Enclosure by MobileApp Systems.}\label{fig:ProtoCase0}
\end{figure}

%---------------------------------------------------------------------
%           CCTV
%---------------------------------------------------------------------
\subsection{CCTV Weatherproof Enclosure}

A strong contender was a generic 15 in. length CCTV Weatherproof Enclosure. This enclosure is primarily aluminum is sealed with heavy duty rubber to prevent water intake. Priced around \$25, this enclosure is an improvement from the Proto Armour. This candidate, however, was rejected for several reasons:

\begin{singlespace}
\begin{enumerate}
    \item Enclosure is larger and heavier than necessary for the system, making inconspicuous installation unlikely.
    \item The large viewing window and the distance where the camera is mounted increases light pollution from the surrounding environment as well as the RPi2 diagnostic LEDs.
    \item LEDs inside: The viewing window isn't wide enough for both the camera module and Night Vision board.
    \item LEDs outside: A secondary enclosure will be required to custom mount onto the CCTV enclosure.
    \item While not required, additional enclosures for the RPi2 and the camera module would be recommended, thereby increasing the cost to the level of the Proto Armour Enclosure.
\end{enumerate}
\end{singlespace}

\begin{figure}[H]
    \centering
    \begin{subfigure}[b]{0.475\textwidth}
        \includegraphics[width=\textwidth]{pics/Case_CCTV1.jpg}
        \caption{External view.}
        \label{fig:CCTVCase1}
    \end{subfigure}
    ~ %add desired spacing between images, e. g. ~, \quad, \qquad, \hfill etc. 
      %(or a blank line to force the subfigure onto a new line)
    \begin{subfigure}[b]{0.475\textwidth}
        \includegraphics[width=\textwidth]{pics/Case_CCTV2.jpg}
        \caption{Internal view.}
        \label{fig:CCTVCase2}
    \end{subfigure}
    \caption{Example CCTV enclosure re-purposed for a RPi2-based security system. Haldas, Mike. \href{https://plus.google.com/u/0/photos/+MikeHaldas/albums/6086069369593277953}{Raspberry Pi Outdoor Camera Housing.}}\label{fig:CCTVCase0}
\end{figure}

%---------------------------------------------------------------------
%           3D PRINT: RPI2
%---------------------------------------------------------------------
\subsection{Main System Enclosure}

After careful consideration, a unanimous decision to design custom weatherproof enclosures was made based on several reasons:

\begin{singlespace}
\begin{enumerate}
    \item A custom design would keep expenditure for enclosures within reasonable bounds. 
    \item The enclosure can be designed so that the gap will be minimal between the camera lens and the enclosure's protective glass cover.
    \item Separate enclosures can be designed: One for the RPi2 and one for the Night Vision board, allowing for modularity through various configurations (for example: two Night Vision Boards).
\end{enumerate}
\end{singlespace}

\begin{figure}[H]
    \centering
    \makebox[\textwidth][c]{\includegraphics[width=1\textwidth]{pics/subsystem.PNG}}
    \caption{Various attachment configurations.}
    \label{fig:caseConfig}
\end{figure}

%\begin{figure}[H]
%    \centering
%    \begin{subfigure}[b]{0.475\textwidth}
%        \includegraphics[width=\textwidth]{pics/Case_RPi_Assembly.PNG}
%        \caption{Assembled view.}
%        \label{fig:RPiCase1}
%    \end{subfigure}
%    ~ %add desired spacing between images, e. g. ~, \quad, \qquad, \hfill etc. 
%      %(or a blank line to force the subfigure onto a new line)
%    \begin{subfigure}[b]{0.475\textwidth}
%        \includegraphics[width=\textwidth]{pics/Case_RPi_Exploded.PNG}
%        \caption{Exploded view.}
%        \label{fig:RPiCase2}
%    \end{subfigure}
%    \caption{CAD Model of the RPi2 outdoors enclosure.}\label{fig:RpiCase0}
%\end{figure}

The enclosures can be 3D printed for quick evaluation. The only possible issue with this method is sourcing the necessary o-rings and screws, but this won't likely be a real concern. Custom designing an enclosure for the Night Vision board improves the functionality of the Home Surveillance System during times low-ambient lighting. By focusing on modularity as a feature, each Nigh Vision enclosure will be able to mount on the top, left, and right sides of the RPi2 enclosure.


%---------------------------------------------------------------------
%           3D PRINT: NIGHT VISION
%---------------------------------------------------------------------
%\subsection{Night Vision Enclosure}

%Custom designing an enclosure for the Night Vision board improves the functionality of the Home Surveillance System during times low-ambient lighting. By focusing on modularity as a feature, each Nigh Vision enclosure will be able to mount on the top, left, and right sides of the RPi2 enclosure.

%\begin{figure}[H]
%    \centering
%    \begin{subfigure}[b]{0.475\textwidth}
%        \includegraphics[width=\textwidth]{pics/Case_RPi_Assembly.PNG}
%        \caption{Assembled view.}
%        \label{fig:LEDCase1}
%    \end{subfigure}
%    ~ %add desired spacing between images, e. g. ~, \quad, \qquad, \hfill etc. 
%      %(or a blank line to force the subfigure onto a new line)
%    \begin{subfigure}[b]{0.475\textwidth}
%        \includegraphics[width=\textwidth]{pics/Case_RPi_Exploded.PNG}
%        \caption{Exploded view.}
%        \label{fig:LEDCase2}
%    \end{subfigure}
%    \caption{CAD Model of the Night Vision outdoors enclosure.}\label{fig:LEDCase0}
%\end{figure}

\newpage
\addcontentsline{toc}{section}{\nameref{sec:challenge}}
\section{Challenges and Considerations}
\label{sec:challenge}

%---------------------------------------------------------------------
%           CURRENT DRAW
%---------------------------------------------------------------------
\subsection{Power and USB Current Draw}

It is recommended to power the system with a 5V, 2A microUSB cable power supply. Using a power supply with lower current capabilities will result in instability.

\subsubsection{USB Hub}
In addition to the 2A power supply consideration, end users that plan on attaching unrelated USB devices could potentially experience intermittent wifi connection, amongst other issues. Current draw from various USB devices was examined through the \verb|usb-devices| and \verb|lsusb -v| commands in Raspbian:

\begin{table}[h]
\centering
\caption{Example average current draw of the Home Surveillance System with unrelated USB devices attached to the RPi2 unit.}\label{tab:current}
\begin{tabular}{@{}lr@{}}
\toprule
Device                      & Current Draw      \\ \midrule
Raspberry Pi                & 750mA             \\
Camera Module               & 100mA--260mA      \\
Wifi Dongle                 & 50mA--100mA       \\
Keyboard and Mouse Combo    & 98mA              \\
External Hard Drive         & 750mA             \\\bottomrule
\end{tabular}
\end{table}

While it might seem okay to run the RPi2 off of a 5V, 2A power supply, this might not be the case depending on additional USB devices unrelated to the home surveillance system. For example, a user could attach some or all of the USB devices listed in Table \ref{tab:current}. The result is an average 1.96A current draw. Without a powered hub, this setup would risk wifi stability when the user leverages the VNC server. A powered USB hub isn't necessary at the current design iteration, however, it is advised to consider one if there are plans to run other USB-based, current-hungry devices on the RPi2 unit.

%---------------------------------------------------------------------
%           CURRENT LIMITER
%---------------------------------------------------------------------
\subsubsection{USB Current Limiter}
The RPi2 includes a current limiter for USB peripherals to reduce brown-outs on the main supply. The total current drawn by USB devices defaults to 600mA, but this can be increased to 1.2A, doubling the default allowance. The RPi2 proof of concept unit \verb|scrying| has been adjusted for this increase. This adjustment is found in \verb|/boot/config.txt|:

\begin{centerverbatim}
max_usb_current=1
\end{centerverbatim}

Increasing this current limit seems to result in a more stable wireless connection for remote viewing through VNC. If power-related issues occur (as indicated by the flickering red power LED on the RPi2), commenting the line out of the file should be the first step in debugging the issue.

%---------------------------------------------------------------------
%           SECURITY
%---------------------------------------------------------------------
\subsection{Security}

The current design relies on VNC to access the RPi2 desktop remotely. The initial process to VNC from an outside network required port forwarding to a specific TCP port. This method presents a security vulnerability; by default, the transmitted data isn't encrypted. This becomes a concern when accessing the system outside of the home network.

Weaved, a cloud-based network platform, was chosen as a safeguard for secure access through VNC and HTTP \cite{weaved}.

\newpage
\addcontentsline{toc}{section}{\nameref{sec:projDev}}
\section{Project Development}
\label{sec:projDev}

%---------------------------------------------------------------------
%           ANTICIPATED FEATURES
%---------------------------------------------------------------------
\subsection{Anticipated Feature: Android App}

An Android app to view the livestream would greatly increase the accessibility of the Home Surveillance System. This feature was tentatively announced during the Proposal Phase, but time constraints limited the resources available for Android development. As a result, the Android app is currently on hold until the next major iteration of the system.

\begin{figure}[H]
    \centering
    \makebox[\textwidth][c]{\includegraphics[width=0.75\textwidth]{pics/android_app.jpg}}
    \caption{Concept UI for the Android app.}
    \label{fig:droidconcept}
\end{figure}

\subsection{Anticipated Feature: Servos}

User-controlled pan-tilt movements is a planned feature for the next major iteration of the Home Surveillance System. The tentative plan is to allow users to control the system through the website livestream as well as through the planned Android app.

%\newpage
%\addcontentsline{toc}{section}{\nameref{sec:directory}}
%\section{Software Directory}
\label{sec:directory}

\begin{center}

\tikzstyle{dir}=[draw=blue,fill=blue!30]
\tikzstyle{every node}=[draw=black,thick,anchor=west]
\tikzstyle{selected}=[draw=red,fill=red!30]
\tikzstyle{optional}=[dashed,fill=gray!50]
\begin{tikzpicture}[%
  grow via three points={one child at (0.5,-0.7) and
  two children at (0.5,-0.7) and (0.5,-1.4)},
  edge from parent path={(\tikzparentnode.south) |- (\tikzchildnode.west)}]
  \node {senior\_design\_spring}
    child { node [dir]{cv\_methods\_isolated}}		
    child { node [dir]{documentation}}
    child { node [dir]{enclosures}}
    child { node [dir] {night\_sensor}
      child { node [dir]{fritzing\_design}}
      child { node {night\_feature.py}}
      child { node {night\_feature\_arduino\_testbed.ino}}
      child { node {night\_feature\_attiny85.ino}}
    }
    child [missing] {}				
    child [missing] {}				
    child [missing] {}
    child [missing] {}	
    child { node {LICENSE}}			
    child { node {README.md}}			
    child { node {texdoc}};
\end{tikzpicture}

\end{center}

\newpage
%\begin{multicols}{2}
%\begin{onehalfspacing}
%\printglossary
%\end{onehalfspacing}
%\end{multicols}

%\newpage
%\section{Appendix}
%\pagenumbering{roman}

%%\begin{multicols}{2}

\newglossaryentry{API}
{
    name=API,
    description={Application Programming Interface is set of tools--usually in the form of a library--to build software-based components and to integrate various technologies.}
}

\newglossaryentry{Blob}
{
    name=blob,
    description={Computer vision method that detects consistent regions within other regions.}
}

\newglossaryentry{Dropbox}
{
    name=Dropbox,
    description={Cloud-based file hosting service that syncs similar-named directories across systems with the configured app.}
}

\newglossaryentry{Google Drive}
{
    name=Google Drive,
    description={Cloud-based file storage and sync service from Google used in this project to draft documents.}
}

\newglossaryentry{GitHub}
{
    name=GitHub,
    description={Online hosting service for git-based source code management and revision control.}
}

\newglossaryentry{Haar}
{
    name=Haar,
    description={Object detection method by P. Viola and M. Jones, \emph{Rapid Object Detection using a Boosted Cascade of Simple Features}.}
}

\newglossaryentry{LAMP}
{
    name=LAMP,
    description={Standard web service stack consisting of open source technologies such as Linux, Apache, MySQL, and Python.}
}

\newglossaryentry{livestream}
{
    name=livestream,
    description={Real-time Internet transmission of an event.}
}

\newglossaryentry{MySQL}
{
    name=MySQL,
    description={My Structured Query Language is a relational database management system released on a GPL v.2 license.}
}

\newglossaryentry{detection}
{
    name=detection,
    description={Techniques to locate objects in still images or video sequences.}
}

\newglossaryentry{recognition}
{
    name=recognition,
    description={Techniques to identify or verify objects in still images or video sequences through a stored database of known faces.}
}

\newglossaryentry{tracking}
{
    name=tracking,
    description={Techniques to locate moving objects in video sequences.}
}

\newglossaryentry{OpenCV}
{
    name=OpenCV,
    description={Open Source Computer Vision is a computer vision library with an open-source BSD license developed at Intel.}
}

\newglossaryentry{open source}
{
    name=open source,
    description={Software and hardware that is released to the public with a free license, allowing for universal access, distribution, and collaborative improvements.}
}

\newglossaryentry{Pi}
{
    name=Raspberry Pi,
    description={The Raspberry Pi is a series of sub-\$50 single-board computers designed for educational usage. For the purposes of this document, all references--unless specified otherwise--refers to the Raspberry Pi 2 Model B.}
}

\newglossaryentry{Plotly}
{
    name=Plotly,
    description={Online data visualization tool with scientific graphing libraries for Python.}
}

\newglossaryentry{REST}
{
    name=REST,
    description={Representational State Transfer is an World Wide Web architectural style for client--server communications.}
}

\newglossaryentry{sighting}
{
    name=sighting,
    description={In this case, each sighting is an interval of time where the system has detected an object.}
}
 
\newglossaryentry{ShareLaTeX}
{
    name=ShareLaTeX,
    description={Online \LaTeX \ editor with revision control and real-time collaboration.}
}
 
\newglossaryentry{SimpleCV}
{
    name=SimpleCV,
    description={Simple Computer Vision is a simplified version of OpenCV. Like OpenCV, SimpleCV is open source.}
}

\newglossaryentry{sprint}
{
    name=sprint,
    description={A basic unit of development in Scrum methodology. It is a timeboxed iteration that contains then existing backlog, current scope of work, and estimated commitment from each resource.}
}

\newglossaryentry{stand-up}
{
    name=stand-up,
    description={A timeboxed meeting at the beginning of the team's workday where each resource gives an update on the events since the last stand-up, the plans for the day, and announces any roadblocks.}
}

\newglossaryentry{VNC}
{
    name=VNC,
    description={Virtual Computing Network is a system used to remotely control another computer.}
}

\newglossaryentry{WebIDE}
{
    name=WebIDE,
    description={Adafruit's web-based Integrated Development Environment designed for the Raspberry Pi.}
}

%\end{multicols}

\addcontentsline{toc}{section}{References}
%---------------------------------------------------------------------
% Bibliography
%---------------------------------------------------------------------
%\addcontentsline{toc}{section}{References}
\begin{singlespacing}
\begin{thebibliography}{20}

%\bibitem{mallat} S. Mallat, ``Fourier Kingdom: Properties'', in \emph{A Wavelet Tour of Signal Processing, 2nd ed.}, Chestnut Hill, MA: Academic Press, 1999.

\bibitem{ViolaJones} P. Viola, M. Jones, ``Robust Real-Time Face Detection,'' in \emph{Int. Jour. Computer Vision}, vol. 57.2, pp.~137-154, 2004.

\bibitem{weaved} ``Raspberry Pi - Weaved Inc'', \emph{Weaved Inc.}, 2016. [Online]. Available: http://www.weaved.com/raspberry-pi-remote-connection/. [Accessed: 14MAR2016].

%\bibitem{haarTri} A. Haselhoff, A. Kummert, ``A vehicle detection system based on Haar and Triangle features'', {Intelligent Vehicles Symposium, 2009 IEEE}, Xi'an, pp.~ 261-266, 2009.

%\bibitem{motionHaar} H. Bai, J. Wu, et al, ``Motion and haar-like features based vehicle detection'', \emph{Multi-Media Modelling Conference Proceedings, 2006 12th Int.}, Beijing, pp.~ 4, 2006.

%\bibitem{vehicleFeatures} R. Kostia, ``Video-based traffic monitoring at day and night time: Vehicle features detection and tracking'', in \emph{Proc. 12th International IEEE Conf.}, pp.~285-290, 2009. 

% http://www.cs.utexas.edu/~grauman/courses/spring2008/slides/Faces_demo.pdf

% http://www.cvip.uofl.edu/wwwcvip/education/ECE523/Spring%202011/Lec7.pdf

% https://www.element14.com/community/community/raspberry-pi/raspberrypi_projects/blog/2014/01/24/pi-noir-with-ir-leds

%\bibitem{meeting} Department of Electrical Engineering and Technology. (2016, Feb 1). ``Lecture Meeting: Daily stand-up for project ELEC4500--2016''. Wentworth Institute of Technology, Boston, MA.

\end{thebibliography}
\end{singlespacing}

\end{document}
